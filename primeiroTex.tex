\documentclass[12pt,a4paper, oneside]{book}

\usepackage[top = 2cm, bottom = 2cm, left = 2.5cm, right = 2.5cm]{geometry}
\linespread{1.5}
\usepackage{setspace}
\usepackage[T1]{fontenc}
\usepackage[brazilian]{babel}
\usepackage{color}
\usepackage[dvipsnames,svgnames,x11names]{xcolor}

\title{Curso de LaTeX}
\author {Rodrigo Camara Barboza \footnote{texto}}
\date{18 de fevereiro de 2021}

\begin{document}
\maketitle

Olá pessoal, tudo bem? Bem vindos a nosso curso de LaTex, aqui você vai aprender muitas coisas legais que irão te prender a atenção e impressionar seus amigos e professores na facul. 

\singlespacing
Olá pessoal, sou eu de novo.

\onehalfspacing
Olá pessoal, tudo bem? Bem vindos a nosso curso de LaTex, aqui você vai aprender muitas coisas legais que irão te prender a atenção e impressionar seus amigos e professores na facul. 

\doublespacing
Olá pessoal, tudo bem? Bem vindos a nosso curso de LaTex, aqui você vai aprender muitas coisas legais que irão te prender a atenção e impressionar seus amigos e professores na facul. \newline

\begin{itemize}
	
	\item[$\sharp$] Olá  % Este ambiente é formado pela inserção de itens de marcadores
						 % Esta linha é um item de compra

\end{itemize}

\chapter{Intodução}

\chapter{Desenvolvimento}

\chapter{Conclusão}

\section{Olá}

\subsection{olá 2}

\subsubsection{olá 3}

\end{document}